\documentclass[twoside]{article}


\usepackage[%paperheight  =210mm,paperwidth   =148mm,  % or: "paper=a5paper"
            paper=a5paper,
            layoutheight =171mm,layoutwidth  = 95mm,
            layoutvoffset= 10mm,layouthoffset= 0mm,
            left=3mm, right=3mm, top=5mm, bottom=5mm,
            includeheadfoot,
            asymmetric,
            bindingoffset=12mm,
            showcrop=true]{geometry}
\usepackage{amsmath}
\usepackage{scrlayer-scrpage}


\makeatletter
\newlength\hcorr
\setlength\hcorr{\paperwidth}
\addtolength\hcorr{-\Gm@layoutwidth}
\makeatother

\geometry{layouthoffset=.5\hcorr,
          layoutvoffset=10mm}

\newcommand\foldmarklength{3mm}
\newcommand\punchmarklength{4mm}
\newcommand\markhpos{3.5mm}
\newcommand\markthickness{.2mm}

\newcommand\leftmarkline[1]{%
  
  \parbox[c][\layerheight][b]{\layerwidth}{%
   \hspace*{.5\hcorr}\hspace*{\markhpos}\rule{#1}{\markthickness}%
}}
\newif\ifFoldmark\Foldmarktrue
\newif\ifPunchmark\Punchmarktrue
\DeclareNewLayer[{
  background,
  innermargin,
  oddpage,% in twoside mode only on odd pages!
  height=10mm+22mm,
  contents={\ifFoldmark\leftmarkline{\foldmarklength}\fi}
}]{firstmark}
\DeclareNewLayer[{
  clone=firstmark,
  height=10mm+22mm+19mm
}]{secondmark}
\DeclareNewLayer[{
  clone=firstmark,
  height=10mm+22mm+19mm+19mm
}]{thirdmark}
\DeclareNewLayer[{
  clone=firstmark,
  height=10mm+22mm+19mm+19mm+51mm
}]{foursmark}
\DeclareNewLayer[{
  clone=firstmark,
  height=10mm+22mm+19mm+19mm+51mm+19mm
}]{fivesmark}
\DeclareNewLayer[{
  clone=firstmark,
  height=10mm+22mm+19mm+19mm+51mm+19mm+19mm
}]{sixedmark}
\DeclareNewLayer[{
  clone=firstmark,
  height=95.5mm,
  contents={\ifPunchmark\leftmarkline{\punchmarklength}\fi}
}]{punchmark}
\AddLayersToPageStyle{@everystyle@}{firstmark,secondmark,thirdmark,foursmark,fivesmark,sixedmark,punchmark}
\title{\LaTeX}
\date{}
\begin{document}
  \maketitle
  \LaTeX{} is a document preparation system for the \TeX{}
  typesetting program. It offers programmable desktop
  publishing features and extensive facilities for
  automating most aspects of typesetting and desktop
  publishing, including numbering and cross-referencing,
  tables and figures, page layout, bibliographies, and
  much more. \LaTeX{} was originally written in 1984 by
  Leslie Lamport and has become the dominant method for
  using \TeX; few people write in plain \TeX{} anymore.
  The current version is \LaTeXe.

  % This is a comment, not shown in final output.
  % The following shows typesetting power of LaTeX:
  \begin{align}
    E_0 &= mc^2                              \\
    E &= \frac{mc^2}{\sqrt{1-\frac{v^2}{c^2}}}
  \end{align}
  \LaTeX{} is a document preparation system for the \TeX{}
  typesetting program. It offers programmable desktop
  publishing features and extensive facilities for
  automating most aspects of typesetting and desktop
  publishing, including numbering and cross-referencing,
  tables and figures, page layout, bibliographies, and
  much more. \LaTeX{} was originally written in 1984 by
  Leslie Lamport and has become the dominant method for
  using \TeX; few people write in plain \TeX{} anymore.
  The current version is \LaTeXe.

  % This is a comment, not shown in final output.
  % The following shows typesetting power of LaTeX:
  \begin{align}
    E_0 &= mc^2                              \\
    E &= \frac{mc^2}{\sqrt{1-\frac{v^2}{c^2}}}
  \end{align}
  \LaTeX{} is a document preparation system for the \TeX{}
  typesetting program. It offers programmable desktop
  publishing features and extensive facilities for
  automating most aspects of typesetting and desktop
  publishing, including numbering and cross-referencing,
  tables and figures, page layout, bibliographies, and
  much more. \LaTeX{} was originally written in 1984 by
  Leslie Lamport and has become the dominant method for
  using \TeX; few people write in plain \TeX{} anymore.
  The current version is \LaTeXe.

  % This is a comment, not shown in final output.
  % The following shows typesetting power of LaTeX:
  \begin{align}
    E_0 &= mc^2                              \\
    E &= \frac{mc^2}{\sqrt{1-\frac{v^2}{c^2}}}
  \end{align}
  \LaTeX{} is a document preparation system for the \TeX{}
  typesetting program. It offers programmable desktop
  publishing features and extensive facilities for
  automating most aspects of typesetting and desktop
  publishing, including numbering and cross-referencing,
  tables and figures, page layout, bibliographies, and
  much more. \LaTeX{} was originally written in 1984 by
  Leslie Lamport and has become the dominant method for
  using \TeX; few people write in plain \TeX{} anymore.
  The current version is \LaTeXe.

  % This is a comment, not shown in final output.
  % The following shows typesetting power of LaTeX:
  \begin{align}
    E_0 &= mc^2                              \\
    E &= \frac{mc^2}{\sqrt{1-\frac{v^2}{c^2}}}
  \end{align}
  \LaTeX{} is a document preparation system for the \TeX{}
  typesetting program. It offers programmable desktop
  publishing features and extensive facilities for
  automating most aspects of typesetting and desktop
  publishing, including numbering and cross-referencing,
  tables and figures, page layout, bibliographies, and
  much more. \LaTeX{} was originally written in 1984 by
  Leslie Lamport and has become the dominant method for
  using \TeX; few people write in plain \TeX{} anymore.
  The current version is \LaTeXe.

  % This is a comment, not shown in final output.
  % The following shows typesetting power of LaTeX:
  \begin{align}
    E_0 &= mc^2                              \\
    E &= \frac{mc^2}{\sqrt{1-\frac{v^2}{c^2}}}
  \end{align}
  \LaTeX{} is a document preparation system for the \TeX{}
  typesetting program. It offers programmable desktop
  publishing features and extensive facilities for
  automating most aspects of typesetting and desktop
  publishing, including numbering and cross-referencing,
  tables and figures, page layout, bibliographies, and
  much more. \LaTeX{} was originally written in 1984 by
  Leslie Lamport and has become the dominant method for
  using \TeX; few people write in plain \TeX{} anymore.
  The current version is \LaTeXe.

  % This is a comment, not shown in final output.
  % The following shows typesetting power of LaTeX:
  \begin{align}
    E_0 &= mc^2                              \\
    E &= \frac{mc^2}{\sqrt{1-\frac{v^2}{c^2}}}
  \end{align}
\end{document}
